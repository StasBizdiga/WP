\section*{Laboratory work \#4}
\phantomsection

\section{Purpose of the laboratory}
Gain knowledge about the basics of working with timer and animation.
\section{Laboratory Work Requirements}
\begin{itemize}
\item \textbf{Basic Level (grade 5 - 6) you should be able to:}
	\begin{enumerate}
	\item Create an animation based on Windows timer which involves at least 5 different drawn objects.
    \end{enumerate}
	
\item \textbf{Normal Level (grade 7 - 8) you should be able to:}
    \begin{enumerate}
    \item Realize the tasks from Basic Level.
    \item Increase and decrease animation speed using mouse wheel/from keyboard
    \item Solve flicking problem describe in your readme/report the way you had implemented this
    \end{enumerate}
          
\item \textbf{Advanced Level (grade 9 - 10) you should be able to:}
    \begin{enumerate} 
    \item Realize the tasks from Normal Level without Basic Level.
    \item Add 2 animated objects which will interact with each other. Balls that have different velocity and moving angles. They should behave based on following rules: 
        \begin{enumerate}
        \item At the beginning you should have 3 balls of different colors of the same size
   	 	\item On interaction with each other, if they are of the same class (circle, square), they should change their color and be multiplied.
    	\item On interaction with the right and left wall (the margins of the window), they should be transformed into squares.
    	\item On interaction with the top and bottom of the window - the figures should increase their velocity.
    	\item Please, take into consideration that the user can increase and decrease animation speed using mouse wheel/from keyboard
		\end{enumerate}
   \end{enumerate}
          
\item \textbf{Bonus point task:}
    \begin{enumerate}
    \item For the task above, add balls with mouse.
    \end{enumerate}
    	
  \end{itemize}  

\clearpage